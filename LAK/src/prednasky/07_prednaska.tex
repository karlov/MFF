\section{\texorpdfstring{Odhady pomoci spektra}{Odhady pomoci spektra}}
\vspace{5mm}
\large

\begin{theorem}[Nezav mnoz v d-regularnim]
	Necht G je d-regularni graf o n vrcholech s vl cisly $\lambda_1 \geq \lambda_2 \geq ... \geq \lambda_n \}$. Pak
	\[ \alpha(G) \leq n \frac{-\lambda_n}{d - \lambda_n} \]
\end{theorem}
\begin{proof}
	Necht A je matice sousednosti grafu G.

	\[ Sp(A) = \{\lambda_1 = d \geq \lambda_2 \geq ... \geq \lambda_n \} \]
	\[ Sp(J) = \{n, 0^{n-1} \} \]

	Matice A, J komutuji $\Rightarrow$ maji spolecnou ortonormalni baze.
	\[ \exists X: X^{\ast}X = E, X^{\ast}AX = \Lambda_A \]
	Kde $\Lambda_A$ je diagonalni matice s vl. cisly na diagonale, rozmíštěné dle uspořadaní. Podobně pro J:
	\[ X^{\ast}AX = \Lambda_J, (\Lambda_J)_{1,1} = n \]

	Z věty o ortonormalní bazi vl. vektor přislušný největšímu vl. číslu je nezaporný. Ostatní maji zaporné složky.
	Pak vektor $\bar{1}$ je prislusny nejvetsimu vlastnimu cislu A - $d$.
	Taky odpovida vl cislu n matice J.

 	Uvazme matici:
	\[ C = A - 1/n(d - \lambda_n)J \]
	Jeji vl. čísla jsou lin. kombinace vl. čísel A, J.
	\[ X^{\ast}CX = X^{\ast}(A - 1/n(d - \lambda_n)J)X = X^{\ast}AX - 1/n(d - \lambda_n)X^{\ast}JX = \Lambda_A - \Lambda_K = \Lambda_C \]
	Kde $(\Lambda_K)_{1,1} = d - \lambda_n$, jinak 0. Z toho $\Lambda_C$ ma na diagonale $\{ \lambda_n, \lambda_2,..., \lambda_n \}$.
	Odtud $\lambda_n$ je největší vl. číslo matice C.

	Necht $W \subseteq V(G)$ je nezav. mnoz G, $|W| = \alpha(G)$. Pak matice A, po seskupeni radku odpovidajicich W, ma nulovou hlavni podmatice odpovidajici W.
	Z toho matice C ma na techto pozicich $1/n(d - \lambda_n)$. Taky je to hlavni podmatice.

	\includegraphics[scale=0.5]{sp_1.eps}

	Vl. cisla matice $-1/n(d - \lambda_n)J$ propletaji vl. cisla matice C.
	\[ Sp(-1/n(d - \lambda_n)J = \{0^{\alpha - 1}, \alpha * -1/n(d - \lambda_n) \} \]
	Z vety o propletani:
	\[ \alpha(G) * -1/n(d - \lambda_n) \geq \lambda_n \Rightarrow \alpha(G) \leq n \frac{-\lambda_n}{d - \lambda_n} \]
\end{proof}

\begin{consequence}
	Necht G je d-regularni graf o n vrcholech s vl cisly $\lambda_1 \geq \lambda_2 \geq ... \geq \lambda_n \}$. Pak
	\[ \chi(G) \geq 1 + \frac{\lambda_1}{|\lambda_n|} \]

	Plyne z toho, ze $ \chi(G) \geq \frac{n}{\alpha(G)}$. Barveni grafu je rozlozeni ne $\chi(G)$ nezavislych mnozin. Kazda z nich ma velikost $\chi(G)/\alpha(G)$. Kombinaci dvou nerovnosti dostaneme tvrzeni.
\end{consequence}

\begin{theorem}[Propletani B]\label{sp_cross_b}
	Necht:
	\[
	\begin{pmatrix}
		A_{11} & A_{12} & ... & A_{1m}\\
		A_{21} & A_{22} & ... & A_{2m}\\
		... & ... & ... & ... \\
		A_{m1} & A_{m2} & ... & A_{mm}\\
	\end{pmatrix}.
\]

	Je Hermitovska matice v blokovem tvaru. $A_{ij} \in \Comp^{m_i \times n_j}$.
	$\sum_{i=1}^m n_i = n$.

	Pak necht $B \in \Comp^{m \times m}$ je matice jejiz prvky $b_{ij} = \frac{\sum_{a \in A_{ij}} a}{n_i} $ jsou prumerne radkove soucty bloky A.
	Potom vl cisla B propletaji vl cisla A.
\end{theorem}
\begin{proof}
	Vezmeme matici $P \in \{0, 1\}^{m \times n}$. Bude rozdelena do bloku velikosti $n_i, i = 1,2,...,n_m$.
	V kazdem radku 1ky jsou v bloku i, jinak nuly.

	\includegraphics[scale=0.6]{sp_2.eps}

	Potom $PP^T$ je diagonalni matice D protoze jednicky jsou na ruznych pozicich. Sk. soucin dvou ruznych radku je 0.
	Na diagonale je norma i-ho radku $= n_i$.

	Pouzijeme matici P abychom dostali radkove soucty matici A:\\
	V matici PA dostaneme sloupcovy soucet po blocich. Pak v matici $PAP^T$ dostaneme soucty vsech prvku v blocich.

	Pro rovnost s matici B jeste potrebujeme vydelit $n_i$. Na coz pouzijeme $D^{-1}$ ktera ma na diagonale $\frac{1}{n_i}$.
	\[ B = D^{-1}PAP^T \]

	Necht $S = D^{-1/2}P$. S je realna matice, pro niz plati
	\[ SS^T = D^{-1/2}PP^T(D^{-1/2})^T = D^{-1/2}DD^{-1/2} = E \]

	Dle Vety o propletani A \cref{sp_cross_a}, vl. cisla $SAS^T$ propletaji vl cisla A.
	\[ SAS^T = D^{-1/2}PAP^T(D^{-1/2})^T = D^{-1/2}DBD^{-1/2} = D^{1/2}BD^{-1/2} \]
	Pak $SAS^T$ a $B$ jsou podobne $\Rightarrow$ maji stejny spektrum.
	\[ Sp(SAS^T) = Sp(B) \]
\end{proof}

\begin{theorem}[Polomer spektra grafu]\label{sp_polomer}
	Necht G je graf o n vrcholech s vl cisly $\lambda_1 \geq \lambda_2 \geq ... \geq \lambda_n \}$. Pak
	\[ \Delta(G) \geq \lambda_1 \geq deg_{avg}(G) \]

	Kde $\Delta(G)$ je max $deg$ grafu.
\end{theorem}
\begin{proof}
	1) Nerovnost $\Delta(G) \geq \lambda_1$. Doplnime G na $\Delta$-regularni graf H tak, aby G byl jeho indukovany podgraf. Pak vl. cisla G propletaji vl cisla H. $\lambda_{max}(H) = \Delta \Rightarrow \Delta(G) \geq \lambda_1$.

	2) Nerovnost $\lambda_1 \geq deg_{avg}(G)$.
	Vezmeme matice sousednosti A, predstavime ji jako matici s 1 blokem. Pak matice prumernych radkovych souctu je $B = deg_{avg}(G)$ jednoprvkova.

	Dle Vety o propletani B \cref{sp_cross_b}, $Sp(B) = \{ deg_{avg}(G)\}$ propleta spektrum A $\Rightarrow \lambda_1 \geq deg_{avg}(G)$.
\end{proof}

\begin{theorem}[Barevnost libovolneho grafu]
	Necht G je graf o n vrcholech s vl cisly $\lambda_1 \geq \lambda_2 \geq ... \geq \lambda_n \}$. Pak
	\[ \chi(G) \leq 1 + \lambda_1 \]
\end{theorem}
\begin{proof}
	Necht H je $\chi$-kriticky indukovany podgraf grafu G. Minimalni stupen vrcholu v $\chi$-kritickem grafu je aspon $\chi - 1$.
	Oznacme jeho největší vl cislo jako $h_1$. Z vety o propletani plyne $\lambda_1 \geq h_1$.
	Z vety polomeru spektra \cref{sp_polomer} dostavame
	\[ h_1 \geq deg_{avg}(H) \geq \delta(H) \geq \chi - 1 \Rightarrow \lambda_1 \geq \chi - 1 \]
\end{proof}

\begin{theorem}[Nezav mnoz v libovolnem grafu]
	Necht G je graf o n vrcholech s vl cisly $\lambda_1 \geq \lambda_2 \geq ... \geq \lambda_n \}$. Pak
	\[ \alpha(G) \leq n \frac{-\lambda_1\lambda_n}{\sigma^2(G) - \lambda_1\lambda_n} \]
\end{theorem}
\begin{proof}
	Necht $W \subseteq V(G)$ je nezav. mnoz G, $|W| = \alpha(G)$. Rozdelime matice A dle W a $V\setminus W$.

	\includegraphics[scale=0.5]{sp_3.eps}

	Pouzijeme Vetu o propletani B \cref{sp_cross_b}.
	\[ B =
	\begin{pmatrix}
		0 & b_{12}\\
		b_{21} & b_{22}\\
	\end{pmatrix}
	\]

	Pak $Sp(B) = \{ h_1 \geq h_2 \}$ propleta $Sp(A) = Sp(G)$.

	Dal vime, ze pocet hran mezi W a $v\setminus W$ se rovna
	\[ \alpha b_{12} = (n - \alpha) b_{21} \Rightarrow b_{21} = \frac{\alpha}{n - \alpha} b_{12}\]

	Z LA soucet vl cisel je determinant:
	\[ h_1h_2 = det(B) = -b_{12} \cdot b_{21} = b_{12}^2 \cdot \frac{\alpha}{n - \alpha}\]

	Z propletani:
	\[ \lambda_1 \geq h_1 \geq h_2 \geq \lambda_n \Rightarrow -h_2 \leq -\lambda_n \Rightarrow -h_1h_2 \leq -\lambda_1 \lambda_n \]

	Protoze vsichni sousede vrcholu z W jsou z $V(G) \setminus W \Rightarrow b_{12} \geq \delta(G)i$.
	\begin{gather*}
	-\delta^2(G) \frac{\alpha}{n - \alpha} \leq -\lambda_1 \lambda_n\\
	-\delta^2(G) \alpha \leq (n - \alpha) * (-\lambda_1 \lambda_n)\\
	\alpha(\delta^2(G) - \lambda_1 \lambda_n) \leq n(-\lambda_1 \lambda_n)\\
	\alpha(G) \leq n \frac{-\lambda_1\lambda_n}{\sigma^2(G) - \lambda_1\lambda_n}
	\end{gather*}
\end{proof}

\begin{theorem}[Propletani A]\label{sp_cross_a}
	Necht $A \in \Comp^{n \times n}$ Hermitovska. $S \in \Comp^{m \times n}$ takova, ze $S^{\ast}S = I$.
	Potom vl cisla $S^{\ast}AS$ propletaji vl cisla matice A.
\end{theorem}
\begin{proof}
	Radky matice S jako vektory v $\Comp^n$ lze rozsirit na ortonormalní baze $\Comp^n$ (Gram-Schmidt z LA). Sestavime z ni matici T, necht
	\[ R = \binom{S}{T} \]
	Pak $RR^{\ast} = I$ a
\[
	RAR^{\ast} =
	\begin{pmatrix}
		SAS^{\ast} & SAT^{\ast} \\
		TAS^{\ast} & TAT^{\ast} \\
	\end{pmatrix}
\]
	Pak $SAS^{\ast}$ je hlavni podmatice $RAR^{\ast}$, a vl cisla $SAS^{\ast}$ propletaji vl cisla $RAR^{\ast}$.
	Pritom $Sp(RAR^{\ast}) = Sp(A)$ z LA, protoze matice jsou podobne.
\end{proof}

\begin{theorem}[Barevnost souvisleho grafu]
	Necht G je souvisly graf o n vrcholech s vl cisly $\lambda_1 \geq \lambda_2 \geq ... \geq \lambda_n \}$. Pak
	\[ \chi(G) \geq 1 + \frac{\lambda_1}{|\lambda_n|} \]
	Veta je analogicka dusledku vety 1, zesiluje ji pro souvisle grafy.
\end{theorem}
\begin{proof}
	Obarvime graf pomoci $\chi$ barev. Necht x je realny vl. vektor prislusny vl. cislu $\lambda_1$ (Existuje dle Frobeniove vety).
	Ze souvislosti $x_i > 0 \forall i$.

	Sestavime matici $P \in \Real^{\chi \times n}$.
	\[ P_{ij} = \twopartdef{ x_j} {j \in W}{0}{j \notin W} \]
	Pak $PP^T = D$ je diagonalni matice, na diagonale $\sum_{u \in W_j}x_u^2 > 0$.
	Necht $S = D^{-1/2}P$.
	Protoze
	\[SS^T = D^{-1/2}PP^T(D^{-1/2})^T = D^{-1/2}DD^{-1/2} = I\]
	Dle Vety o propletani A \cref{sp_cross_a}, vl. cisla $SAS^T$ propletaji vl cisla A.
	Necht vl cisla $SAS^T$ jsou $\{h_1, h_2,..., h_{\chi} \}$. Ma na diagonale same nuly, z toho %todo rozbit dle bloku
	\[ \sum_0^{\chi} h_i = 0 \]

	Dal
	\[ SAS^{T}D^{1/2} \cdot \bar{1} = SAP^TD^{-1/2}D^{1/2} \cdot \bar{1} = SAP^T \bar{1} \]

	\[ P^T \cdot \bar{1} = x \Rightarrow SAP^T \bar{1} = SAx = \lambda_1 Sx = \lambda_1 D^{-1/2}PP^T \bar{1} = \lambda_1 D^{-1/2}D \bar{1} = \lambda_1 D^{1/2} \bar{1} \]
	Dostavame
	\[ SAS^TD^{1/2} \cdot \bar{1} = \lambda_1 D^{1/2} \bar{1} \Rightarrow \lambda_1 \in Sp(SAS^T) \]
	Ale taky odpovida nenulovemu realnemu vl. vektoru, takze $\lambda_1 = h_1$.
	Pouzijeme propletani
	\[ h_1 = \lambda_1 \geq h_2 \geq ... \geq h_{\chi} \geq \lambda_n \land \sum h_i = 0 \Rightarrow -\lambda_1 = h_1 = -\sum_2^{\chi} h_i \]
	Pouzijeme horni odhad pro soucet pres \# scitancu krat min hodnota ($\lambda_n$).
	\[ -\lambda_1 = h_1 = -\sum_2^{\chi} h_i \geq (\chi - 1)(-\lambda_n) \]

	Po uprave
	\[ \chi(G) \geq 1 + \frac{\lambda_1}{|\lambda_n|} \]
\end{proof}
