\section{\texorpdfstring{Sudo-lichomesta, 2-vzdalenost mnozin bodu}{Sudo-lichomesta, 2-vzdalenost mnozin bodu}}
\vspace{5mm}
\large


\begin{lemma}
	$det(S_1 + b_1, S_2 + b_2 , ... , S_k + b_k) = det(S + B), S_i, b_i \in T^k$ kde $S_i,b_i$ jsou sloupce matic $S, B$, jde spocitat jako:
	\[ det(S_1 + b_1, S_2 + b_2 , ... , S_k + b_k) = det(S_1, S_2 + b_2 , ... , S_K + b_K) + det(b_1, S_2 + b_2 , ... , S_K + b_K) \]
	Pak linearita v 2. slozce atd.
	\[ det(S + B) = \sum_{w \subseteq [k]} det(S^wT) \]
	kde $S^w$ znamena, ze jsme vzali sloupce odpovidajici indexum v $w$. Ostatni sloupce jsou z T.
\end{lemma}


\begin{theorem}[skoro dizjunktni systemy mnozin]
	Necht $A_1, ... , A_k$ jsou ruzne $ \subseteq [n]$, $ |A_i \cap A_j| = 1, i \ne j \Rightarrow k \leq n$
\end{theorem}
\begin{proof}
	Necht A-matice incidence $\{A_i\}$. Radek odpovida prvcim, sloupec - mnozinam. Na pozice $(r,s) = 1 \Rightarrow$ prvek r lezi v mnozine $A_s$.

	Vezmeme $A^T * A$ nad $\Real$. Pak ve vysledne matice na pozice $(r,s)$ je $|A_r \cap A_s|$. Jelikoz pruniky jsou 1-prvkove, mame matici 1-cek. Na diagonale jsou $|A_i|$ velikosti mnozin.

	\[ k = rank(A^TA) \leq rank A \leq n \Rightarrow k \leq n \]

	Tvrdime, ze $det(A^TA \ne 0)$. Pak matice je regularni a $rank = k$.
	BUNO $|A_i| = a_i, a_1 \leq a_2 \leq ... \leq a_k$. Mame matici, kde na diagonale jsou velikosti mnozin, jinak 1.

	Nahledneme $a_2 \geq 2$. Jinak pokud $a_1 = a_2 \Rightarrow \exists x \in A_1 \cap A_2 \Rightarrow A_1 = A_2 = \{ x \} $.

	Necht J je matice jednicek. Matici A muzeme napsat jako $J + I*(a_i - 1)$ kde $(a_i - 1)$ je na diagonale.
	%todo link to lemma
	Pouzijeme vlastnost det jako multilinearni formy, viz lemma. Pokud vezmeme 2 sloupce z J, tak det bude 0. Takze zbyvaji det kde je jeden sloupec z S, zbytek z J.

	\[ det(S + J) = det(S) + \sum_i^k J^iS = \]
	Determinanty matic $J^iS$ kde z J je pouze i-ty sloupec lze spocitat rozvojem dle i-ho radku kde je pouze 1 jednicka.
	\[ = \prod_1^k (a_i - 1) + \prod_2^k (a_i - 1) + \sum_{j=2}^k \frac{\prod_1^k (a_i - 1)}{a_j - 2} \]
	Kde 2. produkt mame protoze $a_1$ se muze rovnat 1, zbytek jsou vetsi. Prvni $\prod$ je $\geq 0$, druhy $\prod > 0$ protoze od $i = 2, a_i \geq 2$. $\sum$ je zlomek kladnych clenu, takze $\sum \geq 0$. Dohromady $ det(J + S) > 0$

\end{proof}

\begin{theorem}[sudo-lichomesta]
	Necht $A_1, ... , A_k$ jsou ruzne $ \subseteq [n]$, $ |A_i| = 1 \mod2 \ \forall i, |A_i \cap A_j| \equiv 0 \mod2, i \ne j \Rightarrow k \leq n$
\end{theorem}
\begin{proof}
	Vezmeme matice incidence jako v predchozi vete. Uvazme matici $A^T * A$ nad $\Z_2$. Pak na diagonale jsou mohutnosti mnozin $ = 1 \mod2$, mimo diagonalu pruniky $= 0 \mod2$. Neboli $A^T * A = I \Rightarrow rank = k$. Pak jako minule:

	\[ k = rank(A^TA) \leq rank A \leq n \Rightarrow k \leq n \]
\end{proof}

\begin{definition}
Mnozina bodu v $\Real^n$ je s-vzdalenostni pokud vzajemne vzdalenostni bodu nabyvaji celkem nejvyse s hodnot.
\end{definition}

\begin{observation}
	1-vzdalenostni mnoziny jsou simplexy. Zobecneni rovnostranneho $\triangle$ do vyssich dimenzi. Indukci dokazeme, ze $m_1(n) = n + 1$. Pri prechodu do vyssi dimenze existuje prace jeden bod ktery muzeme pouzit. Proces podobny kompaktizace topologickeho prostoru.
\end{observation}

\begin{theorem}[2-vzdalenostni mnoz]
Necht $m_s(n)$ znaci pocet bodu s-vzdalenostni mnoz v $\Real^n$, pak:
\[ \binom{n+1}{2} \leq m_2(n) \leq 1/2 * (n+1)(n+4) \]
\end{theorem}
\begin{proof}
	1) Dolni odhad

	Vezmeme vektory, ktere maji prave 2 jednicky, jinak 0. Takovych mame $\binom{n}{2}$.

	Pokud 2 vektoru maji 1 spolecnou pozice, $d(x,y) = \sqrt{2}$. Jinak pokud maji 2 spolecne pozice, tak $d(x,y) = 2$. Vzdalenost pocitame jako kanonickou Euklidovou normu.
	\[ m_2(n) \geq \binom{n}{2} \]
	Zesilime dolni odhad: premistime se do $\Real^{n+1}$. Jelikoz $ \sum_i^{n+1} x_i = 2 $, body jsou v nadrovine dimenzi $\Real^n$ kterou lze vzorit do $\Real^n$. Pak:
	\[ m_2(n) \geq \binom{n + 1}{2} \]

	2) Horni odhad

	Mame body $A_1, A_2, ..., A_t$. $A_i = (a_{i,1}, a_{i,2}, ..., a_{i,n}) \in \Real^n$. Oznacme vzdalenosti $k \ne m \in \Real$.

	Definujme funkce $F : \Real^n \times \Real^n \to R, F(x,y) = (d(x,y)^2 - m^2) * (d(x,y)^2 - k^2)$. Pokud je vzdalenost $m \lor k \Rightarrow F = 0$.

	Pak $f_i(x), f_i(x) = F(x, A_i).$ Castecne dosazeni. Tyto funkce jsou v V.P. funkci z $\Real^n$. Tvrdime ze $\{ f_i(x)\}$ jsou LN.
	Pokud dosadime 2 ruzne prvky do $f_i$ tak dostaneme 0 dle definice zobrazeni F. Pro stejny bod $f_i = a^2b^2 \ne 0$.

	\[ \sum_1^t f_i * x_i = 0, x_i \in R, 0 = nulova \ funkce \]
	Podivame se na tuto funkce (lin kombinace funkci) v nejakem bode $A_j$.
	\[ \forall j (\sum_1^t f_i * x_i)(A_j) = \sum_1^t f_i * x_i(A_j) = x_j a^2 b^2 = 0 \Rightarrow x_j = 0 \]
	Neboli funkce jsou LN. Jejich pocet je omezen podprostorem funkci nad $\Real^n$ ve kterem zijou.

	\[ f_i(x) = (d(x,A_i)^2 - m^2) * (d(x,A_i)^2 - k^2) = ((\sum_j^t(x_j - a_{i,j})^2 - a^2)*((\sum_j^t(x_j - a_{i,j})^2 - a^2)\]

	$f_i$ jsou polynomu stupne 4. \# polynomu dle dimenze:
	\begin{enumerate}
		\setcounter{enumii}{-1}
		\item k = 0 konstantni $ = 1$.
		\item k = 1 je n.
		\item k = 2 je $\binom{n}{2}$ pro ruzna $x_i, x_j$ a n pro $x_i^2$.
		\item k = 3 $\binom{n}{3}$ pro ruzna $x_i, x_j, x_k$. Pro $x_i^2x_j = n(n-1)$ a n pro $x_i^2$.
		\item k = 4 podobne
	\end{enumerate}
	Nase funkce jsou z podprostoru polynomu $deg = 4$. Zvolme vhodnou bazi.
	\[ U = \langle 1, x_i, x_i*x_j, x_i^2, (\sum x_j^2)x_i, (\sum x_j^2)^2 \rangle \forall i,j \]

	Dostaneme $dim(U) = 1 + n (lin) + n (kv) + n (kv * lin) + \binom{n}{2} (lin 2) + 1 = 2 + 3n + 1/2 n (n-1) = 1/2 (4 + 5 + n^2)$. Generator $\sum x_j^2$ nepotrebujeme protoze je lin komvinaci $x_j^2$.

\end{proof}
