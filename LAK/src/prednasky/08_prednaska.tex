\section{\texorpdfstring{Shannonova kapacita}{Shannonova kapacita}}
\vspace{5mm}
\large

\begin{definition}
Necht A je abcda, $A = \{a, e, o, h, g\}$.
Pak sestavime graf $G_A = (A, \{xy | x \sim y \})$. Kde ekvivalence znamena, ze x je snadno zameni za y.
\end{definition}
Pak by slo vzit nezavislou mnozinu a pouzivat jen tyto symboly. Zbylo by hodne malo symbolu.

Lepe - dohodneme se na pevne delce. Vezmeme $C \subseteq A^n$. Pak bezpecny kod bude pouzivat pouze slova z C.
Dal sestavime $G_{A^n}$ graf zamenitelnosti pro $A^n$.

\begin{observation}
	2 slova jsou zamenitelna $\iff$ maji na i-te pozice stejne pismeno nebo zamenitelne. Presne odpovida uplneho soucinu grafu.
\end{observation}

\begin{definition}
	Pro grafy G, H definujme uplny soucin grafu jako graf
\[ G \square H = (V(G) \times V(H), \{(a, b)(x, y): (a = x \lor ax \in E(G)) \land (b = y \lor by \in E(H))\}) \]
Kde vrcholy a,x jsou z grafu G, b,y z grafu H.

	Taky definujme $G^n = G \square G \square ... \square G$.
\end{definition}
\begin{definition}
	Shannonova kapacita grafu G:
	\[ \Theta(G) = \sup_k \sqrt[k]{\alpha(G^k)}, \forall k \]
\end{definition}
\begin{observation}
	\[\forall G \Theta(G) \geq \alpha(G)\]

	Pokud v grafu je nezav mnozina $B \subseteq V(G), |B| = \alpha(G)$. Pak $B^k$ je taky nezav mnozina.
	Z toho
	\[ \sqrt[k]{\alpha(G^k)} \geq \sqrt[k]{\alpha(B^k)} = \sqrt[k]{\alpha^k(G)} = \alpha(G) \]
\end{observation}
\begin{observation}
	Necht $\sigma(G) = \chi(-G)$. Coz je minimalni pocet uplnych podgrafu pokryvajici mnoz grafu. Pak
	\[ \Theta(G) \leq \sigma(G) \]
	Protoze $K_n \square K_m = K_mn$. Soucin uplnych je uplny graf, jina moznost neni (jsou tam vsechny hrany).Takze
	\[ \sigma(G^k) \leq \sigma^k(G) \Rightarrow \sqrt[k]{\sigma(G^k)} \leq \sigma(G) \Rightarrow \Theta(G) \leq \sigma(G) \]
\end{observation}
\begin{observation}
	G je perfektni graf $\Rightarrow \sigma(G) = \alpha(G)$. Pak
	\[ \alpha(G) \leq \Theta(G) \leq \sigma(G) = \alpha(G) \]
\end{observation}

\begin{definition}
	Lovascova definice ortonormalni reprezentace grafu je zobrazeni $f:V \to \Real^d$ splnujici:
	\begin{itemize}
		\item $|| f(u) || = \langle f(u), f(u) \rangle = 1 \forall u \in V$ a
		\item $\langle f(a), f(b) \rangle = 0 \forall a \ne b \land ab \notin E(G)$.
	\end{itemize}

	Pak velikost reprezentace je:
	\[ || f || = \inf_{c: ||c|| = 1} \max_{a \in V} \frac{1}{\langle c, f(a) \rangle^2} \]
\end{definition}

\begin{example}
	Pro graf ktery nema zadny vrchol potrebujeme system vzajemne $\perp$ vektoru velikosti $V(G)$, neboli prostor dimenze $V(G)$.

	Pro uplny graf staci volit vektory stejneho smeru nebo dokonce stejne.
\end{example}

\begin{definition}
	Lovascova dzeta funkce grafu G:
	\[ \vartheta(G) = \inf_f || f || \]

	Chceme pro nejakou reprezentace najit takovy jedn vektor c, ktery minimalizuje hodnotu $\langle c, f(u( \rangle^2$.
\end{definition}

\begin{example}
	Pro uplny graf zvolime reprezentaci ktera se sklada ze stejnych vektoru, c vezmeme ve stejnem smeru.
	Pak vsechny skalarni souciny jsou 1. Z toho
	% todo jina funkce, predn 8 od 42:00
	\[ \vartheta(K_n) \leq 1 \]
\end{example}

\begin{definition}
	\emph{Rukojet} reprezentace f je vektor c (jedn vektor), pro ktery f nabyva minima.
	Infimum v def velikosti ortonormalni reprezentace se nabyva, protoze $f = f(c)$ je spojita a zdola omezena.

	V definici staci uvazovat omezenou dimenzi, napr $d \leq |V(G)|$.

	Infimum v dev dzeta funce se taky nabyva, protoze $||f||$ je spojita funkce f. Pak
	\[ \vartheta(G) = \min_f \min_{c: ||c|| = 1} \max_{u \in V} \frac{1}{\langle c, f(a) \rangle^2} \} \]
\end{definition}
\begin{agreement}
	Muze se stat, ze rukojet je vektor kolmy na nejaky z vektoru f. Pak $\vartheta(G) = \infty$.
	Budeme se ale takovym rukojetim vyhybat. Vsechny vektory reprezentace lezi v nadrovine, je jich konecne mnoho.
\end{agreement}

\begin{lemma}
	$\forall G : \alpha(G) \leq \vartheta(G)$.
\end{lemma}
\begin{proof}
	Necht G je graf, a mame optimalni repr. f s rukojeti c. $|| f || = \vartheta(G)$.
	Taky $W \subseteq V(G)$ je nezav mnozina:
	\[ \alpha(G) = |W| \]

	Vektory reprezentujici W jsou na sebe kolme. Muzeme je doplnit na ortonormalni baze B prostoru $\Real^d$.
	Pak rukojet muzeme napsat jako lin. konbinace pomoci vektoru z B:
	\[ c = \sum_{b \in B} \langle c, a \rangle \cdot b \]

	Dal c je jedn. vektor:
	\[ 1 = \langle c, c \rangle = \langle \sum_v \langle c,v \rangle, \sum_v \langle c,v \rangle \rangle = \sum_u \sum_v \langle c,u \rangle \langle c, v \rangle \langle u, v \rangle \]
	vektory u, v jsou z ortonormalni baze, takze pro $u \ne v$ je soucet nula, jinak misto posledniho sk souciny tam bude 1. Pak dostaneme soucet vlevo, ktery je vesti nez suma pro vektory reprezentace nezav mnoziny.
	\[ \sum_{b \in B} \langle c, b \rangle^2 \geq \sum_{u \in W} \langle c, f(u) \rangle^2 \]
	Nahledneme ze velikost sk. soucinu je omezena maximumem pro vsechny vrcholy, coz je prave $\vartheta(G)$.
	\[ \forall a \in V(G): \frac{1}{\langle c, f(a) \rangle^2} \leq \vartheta(G) \Rightarrow \langle c, f(a) \rangle^2 \geq \frac{1}{\vartheta(G)} \]

	\[ \sum_{u \in W} \langle c, f(u) \rangle^2 \geq \sum_{a in W} \frac{1}{\vartheta(G)} \]
	Scitame pres velikost nezavisle mnoziny, dostaneme $ \frac{\alpha(G)}{\vartheta(G)} $
	Dohromady
	\[ 1 = || c || \geq \frac{\alpha(G)}{\vartheta(G)} \Rightarrow \vartheta(G) \geq \alpha(G) \]
\end{proof}

\begin{lemma}
	$\forall G, \forall H: \vartheta(G \square H) \leq \vartheta(G) \cdot \vartheta(H)$. Taky
	\[ \forall G \forall k \in \N: \vartheta(G^k) \leq \vartheta^k(G) \]
\end{lemma}
\begin{proof}
	Necht f je optimalni ortonormalni repr. G s rukojeti c. Podobne g pro H s rukojeti d. Uvazme tenzorovy soucin $f \circ g$ jako ortonormalni reprezentace soucinu grafu.

	\[ (u, v) \in V(G \square H), (f \circ g) (u, v) = (f(u) \circ g(v)) = (f(u)_i g(v)_j)_{i, j}, i = 1, 2, ..., n1; j = 1,2, ..., n2\]
	Vezmeme $(u, v), (u', v'): (uu' \notin E(G) \land u \ne u') \lor (vv' \notin E(H) \land v \ne v') $. Pak

	\[ \langle f(u) \circ g(v), f(u') \circ g(v') \rangle = \langle f(u), f(u') \rangle \cdot \langle g(v), g(v') \rangle \]
	Pak bud jeden sk soucin je 0 nebo druhy z volby vrcholu. Takze
	\[ \langle f(u), f(u') \rangle \cdot \langle g(v), g(v') \rangle = 0 \]

	Pak rukojet pro $G \square H$ bude $c \circ d$. Pak
	\[ || f \circ g || \leq \max_{u, v} \frac{1}{\langle c \circ d, f(u) \circ f(v) \rangle^2} = \max \frac{1}{\langle c, f(u) \rangle^2 \cdot \langle d, g(v) \rangle^2} \]
	Max je dvou funci je mensi nez soucin max dvou funkci:
	\[ \max \frac{1}{\langle c, f(u) \rangle^2 \cdot \langle d, g(v) \rangle^2} \leq \max_u \frac{1}{\langle c, f(u) \rangle^2} \max_v \frac{1}{\langle d, g(v) \rangle^2} = \vartheta(G) * \vartheta(H) \]
\end{proof}

\begin{lemma}
	$\forall G : \Theta(G) \leq \vartheta(G)$.
\end{lemma}
\begin{proof}
	\[ \Theta(G) = \sup_k \sqrt[k]{\alpha(G^k)} \leq \sup_k \sqrt[k]{\vartheta(G^k)} \leq \sup_k \sqrt[k]{\vartheta^k(G)} = \vartheta(G) \]

\end{proof}

\begin{theorem}[Shannonova kapacita $C_5$]
	$ \Theta(C_5) = \sqrt{5}$.
\end{theorem}
\begin{proof}
	Vime $ \alpha(C_5^2) = 5 \Rightarrow \vartheta(C_5) \geq \sqrt{5} $. Ukazeme $\ \vartheta(C_5) \leq \sqrt(5)$. Z toho
	\[ \sqrt{5} \leq \Theta(C_5) \leq \vartheta(C_5) \leq \sqrt{5} \]
	Odkud plati i rovnost.

	Pro dukaz staci uvazit ortonormalni reprezentaci $C_5$ ktera se jmenuje Lovascovuv destnik.
\end{proof}

